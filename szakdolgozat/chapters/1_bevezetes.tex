\Chapter{Bevezetés}

A szakdolgozat a Monopoly játék stratégiáinak vizsgálatára épül, melyeket egy webalkalmazás implementálásával, illetve gépi intelligencia megvalósításával prezentál.

A játék elődjét az 1930-as évek elején Charles Darrow alkotta meg, melynek alapja az \textit{Atlantic City} volt. Eredetileg a család és a barátok szórakoztatására készült esti időtöltésre. Folyamatosan fejlesztette a játékot, hogy még izgalmasabbá tegye azt. Kezdetben a játékcégek nem láttak benne nagy potenciált, 1935-ben azonban Monopoly néven az Egyesült Államokban mégis az egyik legkelendőbb játék lett. Később az egész világon elterjedt, és a mai napig egy nagyon sikeres társasjáték gyerekek és felnőttek körében egyaránt. Azóta többféle verziója is elkészült, különböző témákat feldolgozva (pl.: Star Wars, Trónok Harca), különböző nehézségi szintekkel (pl.: Monopoly Junior).

A pszichológiai szükségleteinkre a társasjátékok jó hatással vannak, mert kiélhetjük bennük társas vágyainkat, a figyelem középpontjába kerülhetünk. Egy-egy játék során újabb és újabb helyzetekben találhatjuk magunkat. Minden ember szeret nyerni, ezért sokan próbálnak nyerő stratégiákat kidolgozni. Mitől függ a nyerés esélye? -- teszik fel sokan a kérdést \cite{shanklin2007using}.

Szakdolgozatomban azt vizsgálom, hogy különböző paraméterek változtatásával befolyásolható-e a győzelem valószínűsége. Ennek megvalósításához gépi intelligenciákat használok, melyeknek működése szintén bemutatásra kerül.

Ezen felül a dolgozatban levezetem a fejlesztés folyamatát az előkészületekkel együttesen, illetve a program folyamatábrája is megtekinthető. Három nagyobb területet fed le tartalmilag a szakdolgozat. Magát a felhasználó által használt kliens felület működését, a gépi játékosok megvalósítását, illetve a szimulációk futtatását és az így kapott eredmények elemzését.
