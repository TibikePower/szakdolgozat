\Chapter{Összegzés}

Szakdolgozatomban a Monopoly játék stratégiáinak vizsgálatát, illetve egy saját webalkalmazás implementációjának megvalósítását tűztem ki célul. Törekedtem arra, hogy egy élvezhető, izgalmas játék legyen, de mégis az eredeti szabályok alapján működjön.

A felhasználói felületet a \textit{VueJS} keretrendszer alkotja \textit{socket.io}-val megvalósított kapcsolattal a NodeJS szerver felé. A szimulációk pedig külön JavaScript fájlokba kerültek, hogy akár többször is lehessen vizsgálódni ugyanazokkal a beállításokkal.

A szakdolgozat megírása alatt rengeteg új helyzettel, problémával szembesültem, amire megoldást kellett keresnem. Ezalatt általam nem ismert technológiákkal ismerkedtem meg. Törekedtem egyedi grafikai elemeket megalkotni, ami máshol még nem látható. Az alap koncepció is erre vonatkozik, miszerint vegyítsük a kondicionáló termet az állatkerttel.

A játékhoz 3, különböző szintű, paraméterezhető gépi intelligenciát implementáltam. Ezek paramétereit külön mérésekkel is vizsgáltam, törekedve arra, hogy végeredményben egy minél jobb stratégiát alkalmazó botot sikerüljön vele létrehoznom.

A program továbbfejlesztésének kapcsán esetlegesen egy publikus szerveren való futtatás is elképzelhető lehet úgy, hogy egyszerre több játék is futhasson egyazon időben. Egy adatbázist is lehetne készíteni köré regisztrációval illetve bejelentkezéssel, aminek segítségével szóba jöhetne akár a mikrotranzakció lehetősége is.  
